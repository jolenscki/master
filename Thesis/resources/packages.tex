\documentclass[%
    fontsize=11pt, % Schriftgröße
    twoside=off % kein einseitiges Layout
]{scrbook} % Dokumentenklasse: KOMA-Script Book
\usepackage{scrlayer-scrpage} % Anpassbare Kopf- und Fußzeilen

\usepackage[utf8]{inputenc} % Textkodierung: UTF-8
\usepackage[T1]{fontenc} % Zeichensatzkodierung

%\usepackage[ngerman]{babel} % Deutsche Lokalisierung
\usepackage{graphicx} % Grafiken

% Schriftart Helvetica:
\usepackage[scaled]{helvet}
\renewcommand{\familydefault}{\sfdefault}

% Silbentrennung:
\usepackage{hyphenat}
\hyphenation{TUM in-te-res-siert} % Eigene Silbentrennung
%\tolerance 2414
%\hbadness 2414
%\emergencystretch 1.5em
%\hfuzz 0.3pt
%\widowpenalty=10000     % Hurenkinder
%\clubpenalty=10000      % Schusterjungen
%\vfuzz \hfuzz

\usepackage[hidelinks]{hyperref} % Hyperlinks
\usepackage[onehalfspacing]{setspace} % 1,5facher Zeilenabstand
\usepackage{calc} % Berechnungen
\usepackage{enumitem} % Mehr Kontrolle über itemize-, enumerate- und description-Umgebungen
\usepackage{relsize} % Schriftgröße in Abhängigkeit von aktueller anpassen
\usepackage{tabularx} % Flexiblere Tabellen
\usepackage{array}    % For new column types
\usepackage{booktabs}
\usepackage{multirow}
\usepackage[table]{xcolor} % For cell color

\usepackage[tablewithout, figurewithout]{caption} % Anpassen von Beschriftungen
\usepackage[colorinlistoftodos,prependcaption,textsize=tiny]{todonotes}
\usepackage{amsthm}
\usepackage{amssymb}
\usepackage{mathtools}
\usepackage{algorithmicx}
\usepackage{algorithm}     % For algorithm floating environment
\usepackage{algpseudocode} % For pseudocode styling
\algnewcommand{\FunctionName}[1]{\textproc{#1}}
\newtheorem{definition}{Definition}
\newtheorem{problem}{Problem}

% \def\wideentry#1#2{\begin{tabular}[t]{l}#1\\\hline#2\\#3\end{tabular}\linebreak[0]\ignorespaces}
\def\wideentry#1#2{\begin{tabular}[t]{l}\hline\textbf{#1}\\\hline#2 \\\hline\end{tabular}\linebreak[0]\ignorespaces}

\newcommand{\cfbox}[2]{%
    \colorlet{currentcolor}{.}%
    {\color{#1}%
    \fbox{\color{currentcolor}#2}}%
}


% Nummerierung von Abbildungen & Tabellen durchgängig, statt nach Kapiteln:
\usepackage{chngcntr}
\counterwithout{figure}{chapter}
\counterwithout{table}{chapter}

% Abkürzungen, Glossare:
\usepackage[%
    xindy,% xindy zum Indexieren verwenden
    acronym,% Separates Akronym-Verzeichnis
    nopostdot,% Kein Punkt am Ende einer Beschreibung im Glossar
]{glossaries}

% Spezielle Befehlsdefinitionen:
\newcommand{\Thema}{}

\usepackage{bookmark} % Lesezeichen

% Unterdrückung layoutbedingter Warnungen
\usepackage[immediate]{silence}
\WarningFilter[layout]{latex}{Reference `LastPage'} % Gesamtseitenzahl
\WarningFilter[layout]{lastpage}{Rerun to get the references right} % Gesamtseitenzahl
\WarningFilter[layout]{latex}{Label(s) may have changed.} % Referenz auf letzte Seite
\WarningFilter[layout]{textpos}{environment textblock* not in vertical mode} % Positionierung Seitenzahl
\WarningFilter[layout]{scrbook}{Change of } % Fußnoten-Trennzeichen im Text
\WarningFilter[layout]{tocbasic}{number width of} % Nummerbreite im Inhaltsverzeichnis
\WarningFilter[layout]{pdfTeX}{name{glo:abk} has been referenced but does not exist, replaced by a fixed one}

% My packages (Joao Olenscki)

% Debugging:
%\DeactivateWarningFilters[layout] % Unterdrückte Warnungen einschalten
%\usepackage{showframe} % Layout-Boxen anzeigen
%\usepackage{layout} % Layout-Informationen
%\usepackage{printlen} % Längenwerte ausgeben
